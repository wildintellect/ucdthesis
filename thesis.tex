% Root File for a UC Dissertation / Thesis
% UCD thesis class: c/o Shwaine <shwaine@shwaine.com>
%
% modified by Dylan Beaudette, 2006,2010
% minor edits and source uploaded by Alex Mandel, 2010
% Source code available at http://github.com/wildintellect/ucdthesis
\documentclass[11pt]{ucdthesis}
%\documentclass[10pt,twoside,final]{ucdthesis}
% \documentclass[10pt,twoside,draft]{ucdthesis}
% \documentclass[10pt,oneside,final]{ucdthesis}

% TODO: this makes strange things happen in the header...
% this is the version that grad studies wants
% \documentclass[11pt,oneside,final]{ucdthesis}

% when we are giving people drafts, use more of the page:
% \usepackage[letterpaper,left=0.75in,right=0.75in,top=1.25in,bottom=0.5in]{geometry}
% need this for the \foreach command
\usepackage{tikz}

% Turn on single spacing with \ssp.
% Turn on double spacing with \dsp.
% By default, your dissertation is double spaced, as is required by UCD.

% spacing in figures and tables and their captions can be
% changed here (\ssp for single-space, empty for same as surrounding
% text); for this to work, the command \figsp has to be included
% in every figure and table right after the \begin{figure}
% \def\figsp{\ssp}
%\def\figsp{}


% useful for drafts
% line numbers:
% http://www.ctan.org/tex-archive/help/Catalogue/entries/lineno.html
%
% \usepackage{lineno}

%
% SVN integration
%
% \usepackage{svnkw}

% customized headers
%
% http://www.ctan.org/tex-archive/help/Catalogue/entries/fancyhdr.html
% \usepackage{fancyhdr}

% a better verbatim environment: c/o Pete Dirac
% use like this:
%   \begin{Verbatim}[fontsize=8]
%       foobar
%    \end{Verbatim}
% \usepackage{fancyvrb}


% more flexible math support
\usepackage{amsmath}

% allow some pages to be landscape
\usepackage{lscape}

% more flexible definition of table environments
\usepackage{ctable}

%need this for \includegraphics{}
\usepackage{graphicx}

%enable the listings package specifically for including programming code
\usepackage{listings}

%Special hack to make code listings not break pages, fyi they must be short then
\usepackage{float}
\floatstyle{plain} % optionally change the style of the new float
\newfloat{Code}{H}{myc}

%test alternative to listings package minted which requires the python pygments package
%minted was installed to latex by hand
%\usepackage{minted}

% TODO: use the new subfig package instead
% http://www.ctan.org/tex-archive/macros/latex/contrib/subfig/
%
%use this to put figures side by side
\usepackage{subfigure}

 % PDF links --- > breaks with some bibliography entries
% \usepackage{hyperref}

% nice looking, parenthetical references
\usepackage[sorting=nyt,natbib=true,citestyle=authoryear,bibstyle=authoryear,maxnames=3,refsection
=chapter]{biblatex}
%\usepackage{chapterbib} % incompatible with biblatex
\bibliography{proposal}
\defbibheading{bibliography}{%
	\section{References}
	}
	
% bibliography can be single-spaced for UC thesis format
\appto{\bibsetup}{\ssp}
	

%make the index
% \usepackage{makeidx}
% \makeindex

% custom colors
\usepackage{color}
% make a color for comments
\definecolor{MyDarkBlue}{rgb}{0,0.08,0.45}

% customized captions with bold label and small, italic text
% table captions are located above tables
% http://www.kronto.org/thesis/tips/custom-captions.html
% http://www.ctan.org/tex-archive/macros/latex/contrib/caption/
% does this have any effect?
%\usepackage{caption}
% \renewcommand{\captionfont}{\small\itshape}
%
\usepackage[hypcap,font=singlespacing]{caption}
\usepackage{subcaption}
% modern method for setting up captions\
\captionsetup{margin=10pt,font=small,labelfont=bf}
%
% fix so that table captions have correct spacing
\captionsetup[table]{position=top}



% %
% %  fit more material on the page:
% %
%
% reset some float-controlling parameters
\renewcommand{\floatpagefraction}{0.8}	% require fuller float pages

% N.B.: floatpagefraction MUST be less than topfraction !!
\renewcommand{\topfraction}{0.9}	% max fraction of floats at top
\renewcommand{\bottomfraction}{0.8}	% max fraction of floats at bottom


% PDF formatting options, indexing, hyperlinking, with control over link style
%Set PDF Metadata
\usepackage[pdftex,
            pdfauthor={Author Name},
            pdftitle={Title Here},
            pdfsubject={Subject},
            pdfkeywords={Comma, List, Keywords},
            %pdfproducer={Latex with hyperref, or other system},
            %pdfcreator={pdflatex, or other tool}
            ]{hyperref}
\hypersetup{
	%driver=pdftex,
	colorlinks=true,
	urlcolor=blue,
	linkcolor=blue,          % color of internal links
    citecolor=blue,        % color of links to bibliography
    filecolor=magenta
}

%Use an additional package to make bookmarks point to the top to tables, figures and listings
\usepackage[all]{hypcap}

%Alex's customizations
\usepackage{indentfirst} %Indents first paragraph of chapter
\usepackage{datatool} %Allows import of csv and other data-tables
\usepackage{varwidth}
\usepackage{color}
\usepackage{rotating}
\AtEveryBibitem{\clearfield{month}} %Cleaner references without month being printed

%%% Document Portion:
\begin{document}


%
%% Title, Front Matter, and Abstract:
% Declarations for Front Matter

% MS Thesis = 0, Phd Dissertation = 1
\isdissertation{0}

% electronic submission? Paper only = 0, Electronic = 1
\iselectronic{1}

\title{Some Title Here}
\author{Author}

% Choices are September, December, March, June
\degreemonth{June}
\degreeyear{2014}
%More examples DOCTOR OF PHILOSOPHY
\degree{Master of Science}

\chair{xxx}
\othermembers{xxx \\
xxx}
\numberofmembers{3}

\prevdegrees{B.S (University of California at Davis) Year}

%Your Graduate Group
\field{xxx}
\campus{Davis}


% add the abstract here

% Their are two abstracts. One that is published externally from your
% dissertation, and one that is internal. Of course, the text of the
% abstract will be the same. So, we define a macro to hold the body of our
% abstract.
% at 345 words - With electronic filing there is no longer a word limit

\newcommand{\myabstract}{
put the abstract here
}


%Not required for electronic submission, you will need to print your other abstract page 2x and hand them in.
% Here is the first, external, abstract.
%\begin{abstract}
%	\myabstract
%	\abstractsignature
%\end{abstract}

\begin{frontmatter}
\maketitle

% A copyright page is optional. If you have one, it must immediately
% follow the title page. For more information about the copyright page
% see the UCD's Office of Graduate Studies web site.
% \copyrightpage

% dedication (optional), remove comment markers to use 
%\begin{dedication}
%\null\vfil
%{\large
%\begin{center}
%xxxx
%\end{center}}
%\vfil\null
%\end{dedication}



\tableofcontents
\listoffigures
\listoftables

% Here is the second, internal, abstract.
% Update: Melissa Danforth 2006
% Inline abstract is now part of front matter according to coordinator
    \newpage
    \begin{inlineabstract}
		%Only enable small if you're trying to make it fit.		
		%\begin{small}
		\myabstract
		%\end{small}
		
    \end{inlineabstract}

%Acknowledgments (optional)
\begin{acknowledgments}
xxxxxxx
\end{acknowledgments}

%Preface (optional)
\chapter*{Preface}
\addcontentsline{toc}{chapter}{Preface}
\input{preface}


\end{frontmatter}


%
% the chapters
%

% set page style:
% make the chapter and section smaller, chapter and section numbers are removed
% fancyplain will keep the page numbers at the bottom of all pages
%\pagestyle{fancyplain} %Note the \fancyplain command !!!
%\renewcommand{\chaptermark}[1]{\markboth{\small{#1}}{}}
%\renewcommand{\sectionmark}[1]{\markright{\small{#1}}{}}


% TODO: this is only for draft copies !!
% start line number printing
%
% \linenumbers



% chapter 1
\chapter{Real Title Here}
%This is an example for a chapter, additional chapter can be added in the skeleton-thesis
%To generate the final document run latex, build and quick build commands on the skeleton-thesis file not this one.
\chapter{Chapter Title Here} %This looks for chapter1.tex

% chapter 2
\chapter{Real Title Here}
%This is an example for a chapter, additional chapter can be added in the skeleton-thesis
%To generate the final document run latex, build and quick build commands on the skeleton-thesis file not this one.
%This is chapter 2, the default skeleton thesis expects 2 chapters
\chapter{Chapter Title Here}

Some text, really should put an example figure. %This looks for chapter2.tex



% note that the 'plainnat' style does not allow URL's in the bibtex entry
%
% some ideas here:
% http://bib2web.djvuzone.org/bibtex.html
%

% reset the page style
%\pagestyle{plain}


% To enable this it will need to be added to toc so it's not in a chapter
%\printbibliography[heading=bibliography]

% the appendix:
% there are several sections, that don't really fit into the main chapters
%
\part*{\addcontentsline{toc}{part}{Appendices}Appendices}
\appendix

% reset page style to fancy
%\pagestyle{fancyplain}

 \input{appendix}


\end{document}
